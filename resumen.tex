\chapter*{Resumen}
La idea de este TFG me surgió cuando le pregunté al profesor que impartía la asignatura Diseño y Pruebas 1 si podía hacer para complementar el juego un jugador automático, y fue él mismo el que me sugirió que sería más indicado para un TFG. Al ser un trabajo pensado para 6 que realizamos entre aproximadamente 3,5 miembros, quedó bastante chapucero, así que empecé a mejorarlo por mi cuenta, hasta que mi padre me dijo que me veía muy entusiasmado con el proyecto, que intentase convertirlo en TFG, y recordé las palabras del profesor. Así, los objetivos de este TFG son el de mejorar el juego para que sea lo más eficiente posible, añadir un jugador automático, mejorar la experiencia de uso general y corregir varios de sus fallos. Además, derivado del framework que se trataba en Diseño y Pruebas 2, también quiero hacer un proyecto base para reducir y automatizar la creación de elementos repetitivos que siempre son iguales, como listados o formularios de edición, así como añadirles comprobaciones de seguridad. Estos son los puntos principales, pero el objetivo de verdad es hacer un proyecto de nivel del que me sienta realmente orgulloso y me acerque más al día en que pueda decir que soy ingeniero informático de verdad porque haya adquirido conocimientos en las diversas áreas que abarca esta carrera.